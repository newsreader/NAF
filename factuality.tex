\section{Factuality}
\label{sec:factuality}

Information about the veracity or factuality of events can be represented in the factuality layer. The information stored in this layer is relevant for finding out whether something happened or not according to a source. It includes information on whether something is certain, probable or possible, it is true or false and whether it is a statement about the future (implying speculation).

The  information is represented pointing to ontologies and specific values, which provides flexibility to add different kinds of information related to factuality and to include information from various modules trained on differently annotated datasets. In the example below, we use the values of factbank and the NewsReader annotations.

The top element of the layer is {\tt <factualities>}. It contains {\tt <factuality>} elements for which we have information concerning their factuality. The \texttt{<factuality>} element has the following attribute and subelements:

\begin{itemize}
\item id (\textbf{required}): the identifier of the factuality element. It takes the prefix ``f''.
\item one {\tt <span>} element (\textbf{required}), which points to the
  terms the factuality has scope over (see Section
  \ref{sec:span}). Typically, the terms will correspond to the
  terms making up a predicate which may be accompanied by one or more of its
  roles.
\item one ore more {\tt <factVal>} that provide information about the factuality of the expression in the span.
\end{itemize}

The {\tt <factVal>} element has the following attributes:

\begin{itemize}
\item \texttt{value} (\textbf{obligatory}): indicates the output of the factuality module
\item \texttt{resource} (\textbf{obligatory}): points to the ontology or theory that introduced the value
\item \texttt{confidence} (optional): indicates the confidence of the module (if provided)
\item \texttt{source} (optional): indicates the module that generated the result (if more than one module was used, else this information can be obtained unambiguously from the NAF header).
\end{itemize}

\begin{Verbatim}[fontsize=\small]
<factualities>
  <factuality id="f1">
    <span>
        <target id="t3"/>
        <target id="t4"/>
        <target id="t5"/>
        <target id="t6"/>
        <target id="t7"/>
    </span>
    <factVal value="CT+" resource="FactBank" 
             confidence="0.83"/>
    <factVal value="CERTAIN" resource="nwr:AttributionCertainty" 
             confidence="0.79"/>
    <factVal value="PROBABLE" resource="nwr:AttributionCertainty"
             confidence="0.11"/>
    <factVal value="NONFUTURE" resource="nwr:AttributionTime"
             confidence="0.91"/>
    <factVal value="POS" resource="nwr:AttributionPolarity"
             confidence="0.67"/>
  </factuality>
</factualities>
\end{Verbatim}

%%% Local Variables: 
%%% mode: latex
%%% TeX-master: "naf"
%%% End: 

