\section{Markable layer}
\label{sec:markable-layer}

NAF includes a ``markable'' layer, whose purpose is to group tokens and
attach information to them. The layer is repsesented under a
\texttt{<markables>} element. This element has the following attribute:
\begin{itemize}
\item \texttt{source}: the source (or purpose) of the layer.
\end{itemize}

There can be many \texttt{<markable>} layers within NAF, and ideally each
one will have a different \texttt{source} attribute.

The \texttt{<markable>} layer comprises one or more \texttt{<mark>}
elements, one per token group. The \texttt{<mark>} element has one mandatory
attribute:
\begin{itemize}
\item \texttt{id} (\textbf{required}): unique identifier starting with the
  prefix ``m''.
\end{itemize}

Apart from this, \texttt{<mark>} element has the same attributes and
sub-elements as the \texttt{<tern>} element described in section
\ref{sec:terms}\footnote{Except that \texttt{<mark>} elements do not contain
  component sub-elements.}.

Below is an example of a markable layer produced by a tool which links token
groups to DBpedia entities:

\begin{Verbatim}[fontsize=\small]
<markables source="DBpedia">
  <!--Football Championship Subdivision-->
  <mark id="m42" lemma="Football Championship Subdivision">
    <span>
      <target id="w128"/>
      <target id="w129"/>
      <target id="w130"/>
    </span>
    <externalReferences>
      <externalRef resource="spotlight"
      reference="http://dbpedia.org/resource/Division_I_(NCAA)"
      confidence="1.0"/>
    </externalReferences>
 </mark>
</markables>
\end{Verbatim}

%%% Local Variables: 
%%% mode: latex
%%% TeX-master: "naf"
%%% End: 
