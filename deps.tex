
\section{Dependency relations}
\label{sec:dependency-relations}

Dependencies represent dependency relations among terms. Each dependency is
represented by an empty \texttt{<dep>} element and span previous terms.\\

The \texttt{<dep>} element has the following attributes:
\begin{itemize}
\item \texttt{from} (\textbf{required}): term id of the source element
\item \texttt{to} (\textbf{required}): term id of the target element
\item \texttt{rfunc} (\textbf{required}): relational function. Among others,
  it can be one of:
  \begin{itemize}
  \item \texttt{mod}: indicates the word introducing the dependent in a
    head- modifier relation.  For instance:

    \begin{tabular}{ll}
      \texttt{mod(by,gift,Peter)} & the gift of a book by Peter\\
      \texttt{mod(of,examination,patient)} & the examination of the patient
    \end{tabular}

  \item \texttt{subj}: indicates the subject in the grammatical relation
    Subject-Predicate. For instance:

    \begin{tabular}{ll}
      \texttt{subj(arrive,John,\_)} & John arrived in Paris\\
      \texttt{subj(employ,Microsoft,\_)} &  Microsoft employed 10 C programmers\\
      \texttt{subj(employ,Paul,obj)} & Paul was employed by Microsoft
    \end{tabular}

  \item \texttt{csubj, xsubj, ncsubj}: The Grammatical Relations (RL) s
    csubj and xsubj may be used for clausal subjects, controlled from
    within, or without, respectively. ncsubj is a non-clausal subject. For
    instance:

    \begin{tabular}{ll}
      \texttt{xsubj(win,require,\_)} &  to win the America's Cup requires heaps of cash
    \end{tabular}

  \item \texttt{dobj}: Indicates the object in the grammatical relation
    between a predicate and its direct object. For instance:

    \begin{tabular}{ll}
      \texttt{dobj(read,book,\_)} & read books\\
    \end{tabular}

  \item \texttt{iobj}: The relation between a predicate and a non-clausal
    complement introduced by a preposition; type indicates the preposition
    introducing the dependent. For instance:

    \begin{tabular}{ll}
      \texttt{iobj(in,arrive,Spain)} &  arrive in Spain\\
      \texttt{iobj(into,put,box)} &  put the tools into the box\\
      \texttt{iobj(to,give,poor)} &  give to the poor
    \end{tabular}

  \item \texttt{obj2}: The relation between a predicate and the second
    non-clausal complement in ditransitive constructions. For instance:

    \begin{tabular}{ll}
      \texttt{obj2(head,dependent)} & \\
      \texttt{obj2(give,present)} &  give Mary a present\\
      \texttt{obj2(mail,contract)} & mail Paul the contract
    \end{tabular}
  \end{itemize}

\item \texttt{case} (optional): declension case
\end{itemize}

Example of dependency relation annotations:
\begin{verbatim}[fontsize=\small]
  <deps>
  <!-- subj(teach, John) -->
  <dep from="t1" to="t2" rfunc="subj" />
  <!-- dobj(teach, Mathematics) -->
  <dep from="t3" to="t2" rfunc="dobj" />
  <!-- iobj(teach, New_York) -->
  <dep from="t8" to="t2" rfunc="iobj" />
  </deps>
\end{verbatim}


%%% Local Variables: 
%%% mode: latex
%%% TeX-master: "naf"
%%% End: 
