\section{Introduction}
\label{sec:introduction}

This document presents the first draft of NAF: NLP Annotation Format
to be used within the NewsReader project. This version of NAF evolves for
the format used in Kyoto, described in \cite{KAF}.


% @@
% [ASF: How about ``NLP Annotation Format'': since we're hoping NAF will be used beyond the project]

% @@TODO: add text. add references. add TAF, NIF, SEM\\

The following properties and desiderata are used as guidelines for defining NAF:

\begin{enumerate}
\item NAF should properly represent linguistic information focusing on two
  kind of linguistic processes (LPs):
  \begin{enumerate}
  \item within document processing: LPs whose granularity is the document
  \item cross document processing, for (event) coreference, etc.
  \end{enumerate}
  \item NAF should be simple
  \item NAF should work for existing NLP modules developed by the partners in NewsReader, i.e. it should be easy and little afford to adapt existing tools to use NAF.
  \item All elements in NAF will be identified with URIs (not document/XML-object internal ids)
  \item NAF should be flexible so that it can contain additional information and alternative representations:
  \begin{enumerate}
  \item It should be possible (and preferably easy) to integrate alternative modules (that may be developed by third parties) in the pipeline
  \item It should be possible to represent other RDF-based layers that link to the URIs used in the SEM annotation layer or to background knowledge
  \end{enumerate}
\end{enumerate}

The general approach for creating NAF will be to start with KAF, which already supports a number of desiderata mentioned above.
An overview of properties taken from KAF and proposed changes is given below:

\begin{enumerate}
  \item NAF will follow the stand-off/multi-layer architecture as also used in related formats such as KAF
  \item NAF will be presented in XML, using the KAF schema as a starting point
  \item The current ids in KAF will be converted to URIs
  \item Additional elements may be added, for instance, to allow for references to the SEM layer or background knowledge
  \item Elements may need to be reordered to turn KAF into a proper RDF graph
\end{enumerate}

% There have been numerous attempts to standardize some aspect of natural
% language processing. To date, the focus of standards (in various stages of
% development) includes morphosyntactic annotation (MAF) [3], syntactic
% annotation (SynAF) [4], and semantic annotation (e.g. SemAF [5]). The
% beforementioned standards concentrate on a specific stage of annotation. The
% two meta-models present different degrees of maturity; MAF has entered the
% last stages of the ISO process, whereas SynAF is at the level of Working
% Draft standard.

% A problem for these formats is that they are difficult to combine. For
% instance, we might want to do both syntactic annotation and semantic
% annotation, and integrate the results. The Linguistic Annotation Framework
% (LAF) [6] is an ISO standard proposal of a data model for linguistic
% annotation. It allows individual annotations within the annotation framework
% to refer to each other, so that the result is a combined analysis of the
% source text.

% Rather than a data model, our aim is a layered annotation format, where
% several processes can add information without losing anything which is
% produced by any previous process. NAF provides annotation layers for basic
% natural language processing and is open to extensions with other annotation
% layers needed by specific applications, which may be standardized later
% on. NAF is compatible with LAF but imposes a more specific standardization
% of the annotation format itself.

% NAF data format has been inspired by standard specifications available in
% the field of Language Resources. Basic motivations for that were to ensure
% intra- and inter-operability and portability. MAF and SynAF were
% investigated as far linguistic annotation for morpho-syntactic and syntactic
% information, respectively, is concerned.

% NAF can be seen as a three-layer format for text annotation: the first two
% layers, explicitly dedicated to representing morphosyntactic and syntactic
% information, are inspired by MAF and SynAF and are implemented "over" the
% semantic layer. For semantic annotation, the ISO community provides SemAF
% which is especially dedicated to the representation of events and time. We
% decided to boost semantic annotation and devised a dialect of the ISO
% standards, where semantic notation is tailored to the specific purposes of
% the project. NAF layers are to be seen as dialects of the ISO standards, yet
% maintaining (different degrees of) mappability to them. Therefore, NAF does
% not corrupt the compliance with ISO standards and their underlying
% philosophy; instead, it is in line with the strategy in ISO which provides
% high-level models (meta-models) able to be adapted, tailored and implemented
% according to specific needs.

NAF comprises several annotations over a text at different linguistic levels
(morphosyntactic, syntactic, semantic) and adopts a stand off strategy for
annotating the source text. The following overall rules are followed in all
layers:

\begin{itemize}
\item \texttt{<span>} elements are used for grouping linguistic elements.
\item Linguistic annotations of a particular level always span elements of previous levels.
\item Linguistic annotations of different levels are not mixed.
\end{itemize}


%%% Local Variables: 
%%% mode: latex
%%% TeX-master: "naf"
%%% End: 
